
\documentclass[12pt]{article}
\pagenumbering{gobble}
\usepackage[T1]{fontenc}
% Można też użyć UTF-8
\usepackage[utf8]{inputenc}
\usepackage{graphicx}
% Język
\usepackage[polish]{babel}
\usepackage{array, xcolor, lipsum, bibentry}
\usepackage[
top    = 3.2cm,
bottom = 2.50cm,
left   = 3cm,
right  = 3cm]{geometry}


\usepackage{graphicx}
\author{\LARGE Wykład 2}
\title{\bfseries\Huge Analiza numeryczna L 2016/2017}
\date{}

 
\begin{document}

\maketitle
\begin{center}
\section*{\Large Podstawowe pojęcia teorii błędów }
\end{center}

\section*{Błąd względny i bezwzględny:}
$$ x = 1.23456789$$
$$ x' = 1.2345679$$
\linebreak
\textbf{Błąd bezwzględny:} $|x-x'|=10^{-8}|$\\
\linebreak
\textbf{Błąd względny:} $|\frac{x-x'}{x} \approx 0.8*10^{-8}$\\
\linebreak
\section*{Komputerowa reprezentacja liczb:}
\textbf{Liczby całkowite:}
$$l \in Z : l= \pm \sum^n_{i=0} e_i2^i, (e_i \in (0;1], e_n = 1)$$
\linebreak
\textbf{Reprezentacja stałopozycyjna:}
$$|\pm|e_0|e_1|e_2|...|e_n|...|e_d|$$ 
d+1 bitóe na liczbę całkowitą ze znakiem.\\
Nie ma problemu, jeśli n < d.\\
Operacje +,-,*, przy założeniu, że wynik jest reprezentowalny są wykonywane dokładnie.\\
\textbf{Liczby rzeczywiste:}
$$ x \in R\{1,0\}: x= sm2^c, (s \in {+,-}, me \in [\frac{1}{2};1), c \in Z)$$
\linebreak
\textbf{Reprezentacja zmiennopozycyjna}\\
\linebreak
t -- liczba bitów na mantysę\\
d-t -- liczba bitów na cechę\\
d+1 -- liczba bitów na liczbę ze znakiem\\
\linebreak
$$|\pm|e_{-1}|e_{-2}| ..mantysa.. |e_{-t}|..cecha..|$$
Ze względów technicznych mantysę musimy "przybliżać" do \textit{t} cyfr.
$$ m_t = \sum^t_{i=1}e*_{-i}2^{-i}$$
rd(x) -- reprezentacja zmiennopozycyjna liczby x\\
\linebreak
\textbf{Twierdzenie:}
$$(x \neq 0), \frac{rd(x)-x}{x} \leq 2^{-t}$$ 
\textbf{Fakt:} błąd zaokrąglenia mantysy nie przekracza $\frac{1}{2}2^{-t}$
$$|m-m_t| \leq \frac{1}{2} $$
\section*{Zbiór $X_{el}$}
1. $m_t \in [\frac{1}{2};1)$\\
2. $l_max = - l_min = 2^{d-t+1} -1$\\
Czyli możmy reprezentować tylko liczby $x \neq 0$, spełniajace warunek: $\frac{1}{2D} \leq |x| < D$, gdzie $D = 2^{lm-x}$
\textbf{Definicja:} Zbiór liczb zmiennopozycyjnych $X_{fl}$ określamy w następujący sposób:
$$x_{fl} = rd(x)$$
\textbf{Zmiennopozycyjna realizacja działań arytmetycznych:}
$$ x,y \in X_{fl}: fl(x \circ y)= (x \circ y)(1+ \epsilon_0), \circ \in {+,-,*,/}, |\epsilon_0| \leq 2^{-t}$$
$$|\frac{(x\circ y)- fl(x \circ y)}{x \circ y}| = |\epsilon_0|$$
\textbf{Twierdzenie o kumulacji błędów:}\\
Narzędziem pozwalającym w "prosty" sposób analizować realizację zmiennopozycyjną programów komputerowych jest tzw. twierdzenie o kumulacji błędów.\\
\linebreak
Jeżeli $|\sigma_i| \leq 2^{-t}, (1 \leq i \leq n)$, to zachodzi równość:
$$\Pi^n_{i=1} (1+\sigma_i) = 1+ \eta_n$$
gdzie $\eta_n = \sum^n_{i=1} \sigma_i + \theta(2^{-t})$\\
Przy pewnych realistycznych założeniach (związanych z architekturą procesora) zachodzi:
$$| \eta_n| < n2^{-t}$$
\section*{Zjawisko utraty cyfr znaczących}
Zmiennopozycyjne realizacje działań mnożenia i dzielenia uznaje sie za "bezpieczne". Problemy pojawiają się przy odejmowaniu (dodawaniu) liczb, szczególnie tych bliskich sobie.
\pagenumbering{arabic}
\setcounter{page}{2}




\end{document} 
