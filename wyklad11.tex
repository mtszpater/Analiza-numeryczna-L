
\documentclass[12pt]{article}
\pagenumbering{gobble}
\usepackage[T1]{fontenc}
% Można też użyć UTF-8
\usepackage[utf8]{inputenc}
\usepackage{graphicx}
% Język
\usepackage[polish]{babel}
\usepackage{array, xcolor, lipsum, bibentry}
\usepackage[
top    = 3.2cm,
bottom = 2.50cm,
left   = 3cm,
right  = 3cm]{geometry}


\usepackage{graphicx}
\author{\LARGE Wykład 9}
\title{\bfseries\Huge Analiza numeryczna L 2016/2017}
\date{}

 
\begin{document}

\maketitle
\begin{center}
\section*{\Large Kwadratury -- całkowanie numeryczne}

\end{center}
\vspace{5mm}
\textbf{Podstawowe całki:}\\
$$ \int f(x)dx = F(x) + C \Leftrightarrow F'(x) = f(x)  $$
$$ \int x^ndx = \frac{1}{n+1}x^{n+1} + C $$
$$ \int \frac{dx}{x} = ln|x| +C $$
$$ \int sin(x)dx = -cos(x) + C $$
$$ \int \frac{dx}{x^2 + 1} = arctg(x) +C$$
$$ \int xe^{x^2}dx = \frac{1}{2}e^{x^2} + C $$
$$ \int\limits_{a}^{b}f(x)dx = F(b) - F(a) $$
\textbf{Idea całkowania numerycznego:}\\
f - funkcja trudna\\
g - funkcja łatwa (do całkowania), np. wielomian\\
$$ x \in [a,b]: f(x) \approx g(x) \Rightarrow \int\limits_{a}^{b}f(x)dx \approx \int\limits_{a}^{b}g(x)dx $$
\newpage
\section*{Kwadratury liniowe}
\begin{center}
1. $F \equiv F[a,b] $ -- zbiór funkcji całkowalnych w [a,b]\\
2. Funkcjonał I: $I(f) = \int\limits_{a}^{b}f(x)dx$ dla $(f \in F)$
\end{center}
Niech dane będa parami różne liczby $x_0^{(n)}, x_1^{(n)},..., x_n^{(n)} $ \textbf{(węzły kwadratury)} oraz liczby rzeczywiste $A_0^{(n)}, A_1^{(n)},...,A_n^{(n)}$ \textbf{(współczynniki/wagi kwadratury)}. Wyrażenie postaci:
$$ Q_n(f) = \sum_{k=0}^{n} A_k^{(n)}f(x_k^{(n)} $$
nazywamy kwadraturą liniową.\\
\linebreak
\textbf{Cel:} Dobrać współczynniki $A_k^{(n)}$ oraz węzły $x_k^{(n)}$ w taki sposób, aby dla "wielu" $f \in F$ zachodziło:
$$I(f) \approx Q_n(f)$$
Zatem węzły i współczynniki nie powinny być bezpośrednio związane z funkcją f.\\
\linebreak
\textbf{Błąd kwadratury:}\\
$$ I(f) = Q_n(f) + R_n(f)$$,
gdzie $R_n(f)$ to błąd kwadratury.\\
\linebreak
\textbf{Rząd kwadratury:}\\
Mówimy, że kwadratura liniowa $Q_n$ ma rząd \textit{r} wtw, gdy:
$$ \forall_{ w \in \Pi_{r-1}} R_n(w) = 0 \hspace{1cm}(I(w) = Q_n(w))$$ 
$$ \exists_{w \in \Pi_{r}/\Pi_{r-1}} R_n(w) \neq 0 \hspace{1cm}(I(w) \neq Q_n(w))$$  
Przyjmując, że rząd kwadratury jest dobrym wyznacznikiem jej jakości, powinniśmy przy ustalonym n dążyć do zmaksymalizowania rzędu kwadratury.\\
\linebreak
\textbf{Twierdzenie:} Rząd kwadratury liniowej
$$Q_n \le 2n +2 $$
\newpage
\section*{Kwadratury interpolacyjne}
\textbf{Idea:} zastąpić funkcję podcałkową wielomianem interpolacyjnym dla węzłów $x_0^{(n)}, x_1^{(n)},..., x_n^{(n)} $  i całkę $ \int\limits_{a}^{b}f(x)dx$ przybiżyć całkę $\int\limits_{a}^{b}Ln(x)dx$.\\
$$\int\limits_{a}^{b}f(x)dx \approx \int\limits_{a}^{b}Ln(x)dx = \int\limits_{a}^{b}\sum_{k=0}^n (\lambda_k(x)f(x_k))dx$$
$$ \lambda_k(x) = \Pi_{i=0,i \neq k}^{n} \frac{x-x_i}{x_k-x_i}$$
$$\int\limits_{a}^{b}Ln(x)dx = \sum_{k=0}^n (\int\limits_{a}^{b}\lambda_k(x)dx)f(x_k) \equiv \sum_{k=0}^n A_k^{(n)}f(x_k^{(n)} $$
$$ I(f) = \int\limits_{a}^{b}f(x)dx \approx Q_n(f) =  \sum_{k=0}^n A_k^{(n)}f(x_k^{(n)} $$
\linebreak
\textbf{Twierdzenie:} Rząd kwadratury liniowej$ Q_n$ wynosi co najmniej n+1 wtw, gdy jest ona kwadraturą interpolacyjną.\\
Wniosek: 
$$n+1 \le rząd(Q_n) \le 2n+2 $$
\linebreak
\textbf{Błąd kwadratury interpolacyjnej:}\\
Jeśli $f \in C^{n+1}[a,b]$ to błąd interpolacji wyraża się wzorem:
$$ r_n(x) = \frac{f^{(n+1)}\eta(x)}{(n+1)!}(x-x_0)...(x-x_n)$$
$$ \int\limits_{a}^{b}f(x)dx = \int\limits_{a}^{b}Ln(x)dx + \int\limits_{a}^{b}rn(x)dx \equiv Q_n(f) + R_n(f) $$
\newpage
\pagenumbering{arabic}
\setcounter{page}{2}




\end{document} 
