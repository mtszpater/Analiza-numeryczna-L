
\documentclass[12pt]{article}
\pagenumbering{gobble}
\usepackage[T1]{fontenc}
% Można też użyć UTF-8
\usepackage[utf8]{inputenc}
\usepackage{graphicx}
% Język
\usepackage[polish]{babel}
\usepackage{array, xcolor, lipsum, bibentry}
\usepackage[
top    = 3.2cm,
bottom = 2.50cm,
left   = 3cm,
right  = 3cm]{geometry}


\usepackage{graphicx}
\author{\LARGE Wykład 5}
\title{\bfseries\Huge Analiza numeryczna L 2016/2017}
\date{}

 
\begin{document}

\maketitle
\begin{center}
\section*{\Large Postać wielomianu }
\end{center}
$\Pi_n$ -- zbiór wielomianów stopnia $\leq$ n.\\
\section*{Postać naturalna(potęgowa):}
$$ w \in \Pi_n: w(x) = sum^n_{k=0} a_kx^k, (dane: a_0,a_1,..., a_n)$$
\textbf{Schemat Hornera:}\\
\linebreak
$w_1 = a_1$\\
$w_k = w_{k-1}x +a_k, (k = n-1,n-1,...0)$\\
wtedy:\\
$W(x) = w_0$\\
\linebreak
\textbf{Twierdzenie:} Algorytm Hornera jest algorytmem numerycznie poprawnym.\\
\linebreak
\textbf{Twierdzenie:} zachodzi następujące oznaczenie:
$$|\frac{fl(w_0)-w(x)}{w(x)}| \leq E \frac{\sum^n_{k=0}|a_x x^k|}{\sum^n_{k=0}|a_x x^k|}$$
\textbf{Uogólniony schemat Hornera}\\
\linebreak
$w_1 = b_1$\\
$w_k = w_{k+1}(x-x_k) + b_k, (k = n-1,n-1,...,0)$\\
wtedy $w(x) = w_0$\\
\linebreak
\textbf{Twierdzenie:} Uogólniony schemat Hornera jest algorytmem numerycznie poprawnym.
\section*{Postać iloczynowa wielomianu}
\textbf{Definicja: Wielomiany Czebyszewa}
$$
\left\{ \begin{array}{l}
T_0(x) = 1, T_1(x) = x\\
T_k(x) =2xT_{k-1}(x)-T_{k-2}(x), (k=1,2,3,...)\\
\end{array} \right.
$$
\textbf{Własności wielomianu Czebyszewa:}\\
$1^{o} T_n \in \Pi_n \setminus \Pi_{n-1}$, czyli $T_n$ jest wielomian dokładnie n-tego stopnia.\\
$2^{o} T_n(x) = 2^{n-1}x^n+...(n \geq 1)$\\
$3^{o}$ Dla $x in [-1;1]$ mamy:
$$T_n(x) = cos(narcos(x))$$
$4^{o}$
\pagenumbering{arabic}
\setcounter{page}{2}




\end{document} 
