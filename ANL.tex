% Nazwę pliku proszę zmienić na własne imię i nazwisko (bez znaków
% diakrytycznych).
% Kompilować poleceniem pdflatex (sprawdzono w instalacji TeX Live 2014,
% zob. https://www.tug.org/texlive/)

% Możliwe opcje klasy: polish, english, russian — wybór języka eseju.
\documentclass[polish]{kbk}
\usepackage{amsmath}
\begin{document}

\author{Imię i~nazwisko autora}
\title{Analiza numeryczna L 2016/2017}

% Tytuł bez formatowania do żywej paginy. Może być skrócony, jeśli się
% nie mieści. Opcjonalny.
%\titlerunning{}

% Adres e-mail. Opcjonalny.
\email{github.com/anl}

\maketitle

\section{Tytuł wykladu}
\section{Tytuł wykladu}
\section{Tytuł wykladu}
\section{Tytuł wykladu}
\section{Tytuł wykladu}
\section{Tytuł wykladu}
\section{Tytuł wykladu}
\section{Tytuł wykladu}
\section{Tytuł wykladu}



\section{Wielomianowa aproksymacja średnio kwadratowa na zbiorze dyskretny,}

\begin{center}
\textbf{Zadanie}
\end{center}
Ustalmy liczbę naturalną  \(m\) . Niech dana będzie funkcja \(f\) o znanych wartościach dla argumentów parami różnych \(x_0, x_1, ... , x_N\) tzn. 
 \(y_k = f(x_k), (k =0,...,N) \) są dane.\\
Skonstruować taki wielomian \(   w^{*}_m \in \Pi_{m}  \) ~\cite{znaczek}, że:
$$ || f -  w^{*}_m ||_2 = min_{ w_m \in \Pi_m } || f - w_m||_2 = min_{ w_m \in \Pi_m } \sqrt{  \sum_{k=0}^{\infty} (y_k - w_n (x_k))^2 }$$
\\
\\
Pokażemy jak szybko i efektywnie numerycznie znaleźć \(W^{*}_n \) bez konieczności rozwiązywania układów równań normalnych
W tym celu musimy wprowadzić szereg różnych narzędzi.
\\
\begin{center}
\textbf{Definicja - Dyskretny iloczyn skalarny}
\end{center}

Dyskretnym iloczynem skalarnym funkcji \(f\) i \(g\) określony na zbiorze  \(x_0, x_1, ... , x_N\) nazywamy wyrażenie postaci
$$ (f,g)_N =  \sum_{k=0}^{N} f(x_k)  g(x_k) $$


\begin{center}
\textbf{Własności - Dyskretny iloczyn skalarny}
\end{center}
$$ (f,f)_N \geq 0 \quad (f,f)_N = 0  \leftrightarrow  f(x_k) = 0  \quad k = 0,...,N$$
$$ (f,g)_N = (g,f)_N $$
$$ (\alpha f,g)_N = \alpha (f,g)_N $$
$$ (f+h, g)_N = (f,g)_N + (h,g)_N $$
$$ ||f||_2 = \sqrt { (f,f) } $$ 

\begin{center}
\textbf{Definicja - Ortogonalność}
\end{center}
Mówimy, że funkcje \(f,g\) są ortogonalne względem iloczynu skalarnego wtw. gdy \( (f,g) = 0 \)
\textbf{Przykład} \\
$$ (f,g)_N = \sum_{k=0}^{3} f(x_k)g(x_k) \quad  \quad x_0 = 0,  \quad x_1 = \frac{1}{3},  \quad x_2 = \frac{2}{3},  \quad x_3 = 1 $$
$$ f(x) = \cos{2x} $$
$$ g(x) = x^2 - \frac{13}{9}x $$
Sprawdźmy, że \(f,g\) są ortogonalne przy takim wyborze iloczynu skalarnego.
$$ f(x_0)g(x_0) = 0 $$
$$ f(x_1)g(x_1) = \frac{5}{27} $$
$$ f(x_2)g(x_2) = - \frac{7}{27} $$
$$ f(x_3)g(x_3) = - \frac{4}{9} $$

$$(f,g) = 0 + \frac{5}{27}  - \frac{7}{27} - \frac{4}{9} = 0 $$


\begin{center}
\textbf{Definicja - ortogonalny układ funkcji}
\end{center}
Mówimy, że układ funkcji \( (f_0, f_1,..., f_m)  \quad m \in N \) nazywamy układem ortogonalnym względem iloczynu skalarnego jeśli
  \[
    \left\{\begin{array}{lr}
        (f_i, f_j)_n = 0 & i \neq j\\
        (f_i, f_i)_N = 0 \\
        \end{array}\right\} 
  \]

\begin{thebibliography}{99}
\bibitem{znaczek} aby na pewno ten znaczek? xD

\end{thebibliography}

\end{document}
